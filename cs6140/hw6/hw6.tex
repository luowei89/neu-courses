\documentclass[11pt,a4paper,fleqn]{article}
\usepackage[margin=1in]{geometry}
\usepackage{graphicx} 
\usepackage{pgfplots}
\begin{document}
\begin{center}
\textbf{CS6140 Machine Learning Fall 2014 Homework 6, Wei Luo }\\
\end{center}
\textbf{PROBLEM 4}\\
\\
When $\alpha=0$, $y_i(w^Tx_i+b) > 1$, these points are not support vectors.\\
When $0 < \alpha < C$, $y_i(w^Tx_i+b) = 1$, these points are support vectors on the margin line.\\
When $\alpha = C$, $y_i(w^Tx_i+b) < 1$, these points are support vectors inside the margin line. They could be even missclassified.\\
\\ \noindent
\textbf{PROBLEM 5}\\
\\
We can plot the data as follow: red x is for label 0 red o is for label 1\\
\begin{tikzpicture}
\begin{axis}[
    xmin=-0.5, xmax=2.5,
    ymin=-0.5, ymax=2.5,
    axis lines=center,
    axis on top=true,
    domain=-0.5:2.5,
    ]
    \addplot [domain={-0.5:2.5},mark=none,dashed] {1-x};
    \addplot [domain={-0.5:2.5},mark=none,solid] {1.5-x};
    \addplot [domain={-0:2},mark=none,dashed] {2-x};
    \addplot [draw=red,only marks,mark=x] table {
    1 1
    2 2
    2 0  
    };
    \addplot [draw=red,only marks, mark=o] table {
    0 0
    1 0
    0 1
    };
\end{axis}
\end{tikzpicture}\\
The optimal hyperplane is x+y-1=0. Margin is $\sqrt{2}/4$\\
And support vectors are: (0, 1) (1, 0) (1, 1) (2, 0)\\
\end{document}