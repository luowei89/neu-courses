\documentclass[11pt,a4paper,fleqn]{article}
\usepackage[margin=1in]{geometry}
\usepackage{graphicx}
\usepackage{subcaption}
\makeatletter
\setlength{\@fptop}{0pt}
\setlength{\@fpbot}{0pt plus 1fil}
\makeatother
\begin{document}
\begin{center}
\textbf{CS6220 Data Mining Fall 2014 Homework 2, Wei Luo}\\
\end{center}
\textbf{1. Clustering Evaluation}\\ \\
The clusters and ground truth label can be summarized as:\\
\begin{tabular}{|c|c|c|c|c|}
\hline
cluster&1&2&3&4\\
\hline
labels&\{2,2,2,2,2,4\}&\{3,3,3,3,3,1\}&\{1,1,1,4,1\}&\{4,4,4\}\\
\hline
\end{tabular}\\ \\
$purity = \frac{1}{20}*(5+5+4+3)=0.85$\\
$TP = {5 \choose 2} + {5 \choose 2} + {4 \choose 2} + {3 \choose 2} = 29$\\
$FP = 5 +5 + 4 = 14$\\
$FN = 4 + 0 + 0 + 7 = 11$\\
$TN = 5*14+1*10+5*8+1*4+4*3 = 136$\\
precision: $P = \frac{TP}{TP+FP} = \frac{29}{29+14} = 0.6744$\\
recall: $R = \frac{TP}{TP+FN} = \frac{29}{29+11} = 0.725$\\
F-measure: $F_1=\frac{2PR}{P+R} = \frac{2*0.6744*0.725}{0.6744+0.725} = 0.6988$\\
For clusters $C$ and ground true labels $\Omega$:\\
$I(\Omega,C) = \frac{1}{20}\log\frac{20*1}{5*6}+\frac{4}{20}\log\frac{20*4}{5*5}+\frac{5}{20}\log\frac{20*5}{5*6}+\frac{5}{20}\log\frac{20*5}{5*6}$\\
\indent \indent \indent $+\frac{1}{20}\log\frac{20*1}{5*6}+\frac{1}{20}\log\frac{20*1}{5*5}+\frac{3}{20}\log\frac{20*3}{5*3}=0.9909$\\
$H(\Omega) = -(\frac{5}{20}\log\frac{5}{20} +  \frac{5}{20}\log\frac{5}{20} + \frac{5}{20}\log\frac{5}{20}  + \frac{5}{20}\log\frac{5}{20}) = 1.3863$\\
$H(C) = -(\frac{6}{20}\log\frac{6}{20} +  \frac{6}{20}\log\frac{6}{20} + \frac{5}{20}\log\frac{5}{20}  + \frac{3}{20}\log\frac{3}{20}) = 1.3535$\\
Normalized Mutual Information: $NMI(\Omega,C) = \frac{I(\Omega,C)}{\sqrt{H(\Omega)H(C)}}= 0.7234$\\ \\ \\
\textbf{2. Understanding and comparing different clustering algorithms.}\\ \\
(1) run the code with commands like:\\
$>$ python k\_means.py dataset1.txt\\
$>$ python emp.y dataset2.txt\\
$>$ python dbscan.py dataset3.txt 0.3 10\\
Where the first parameter (dataset*.txt) is the file name of the dataset. For DBSCAN, the second parameter is $eps$, the third parameter is $minPts$\\
(2) The purity and NMI of each algorithm for each dataset is:\\ \\
\begin{tabular}{|c|l|l|l|l|}
\hline
&&dataset1.txt&dataset2.txt&dataset3.txt\\
\hline
K-means&Purity&1.000000&0.965000&0.820000\\
 &NMI&1.000000&0.869672&0.319923\\
\hline
DBSCAN&Purity&1.000000&0.965000&1.000000\\
 &NMI&1.000000&0.922051&1.000000\\
\hline
EM&Purity&1.000000&1.000000&0.880000\\
 &NMI&1.000000&1.000000&0.472501\\
\hline
\end{tabular}\\ \\
In DBSCAN algorithm, I used\\
$eps = 0.7, minPts = 15$ for dataset1\\
$eps = 0.9, minPts = 5$ for dataset2\\
$eps = 0.3, minPts = 10$ for dataset3\\ \\
The cluster result as scatter plot is:\\
\begin{figure}[t!]
\begin{subfigure}{.3\textwidth}
\includegraphics[width=\linewidth]{K-means1.png}
\end{subfigure}
\begin{subfigure}{.3\textwidth}
\includegraphics[width=\linewidth]{K-means2.png}
\end{subfigure}
\begin{subfigure}{.3\textwidth}
\includegraphics[width=\linewidth]{K-means3.png}
\end{subfigure}\\
\begin{subfigure}{.3\textwidth}
\includegraphics[width=\linewidth]{DBSCAN1.png}
\end{subfigure}
\begin{subfigure}{.3\textwidth}
\includegraphics[width=\linewidth]{DBSCAN2.png}
\end{subfigure}
\begin{subfigure}{.3\textwidth}
\includegraphics[width=\linewidth]{DBSCAN3.png}
\end{subfigure}\\
\begin{subfigure}{.3\textwidth}
\includegraphics[width=\linewidth]{EM1.png}
\end{subfigure}
\begin{subfigure}{.3\textwidth}
\includegraphics[width=\linewidth]{EM2.png}
\end{subfigure}
\begin{subfigure}{.3\textwidth}
\includegraphics[width=\linewidth]{EM3.png}
\end{subfigure}
\end{figure}
\end{document}